\frontchapter{Preface}

This is a book on Geometry and Trigonometry for high-school students. Usually the books on geometry and
trigonometry are two separate books. However, because the concepts are closely related, I decided to
merge both of them in a single book. Before this I wrote the book on high-school algebra, so it is
natural to cover trigonometry as the next book. Algebra and trignomoetry form bulk of the mathematics
for high-school syllabus, therefore, it is necessary that students get good grasp on the basic concepts
of both of these subjects.

Usually, geometry is spread over 3-4 years of syllabus in schools, however, this book presents all the
material of those years in one book. It covers lines, triangles, quadrailaterals, convex polygons and
circles. Trigonometry will cover plane trigonometry with some information on circles. This is
a book for self study and is not recommended for courses in schools and universities.

Trigonomtery and geometry are most fundamental subjects in Mathematics as further study of subjects like
coordinate geometry, calculus, engineering and rest all depend on it. It is very important to understand
these subjects for the readers if they want to advance further in mathematics.

\section Who should read this book?

Anyone with basic knowledge of class 10 maths should be able to read this book without much difficulty. Other
than that there are no other prerequisite. Since this book is written for self study anyone with interest in
trigonometry and geometry can read it. That does not mean that school or college students cannot read
it. You need to be selective as to what you need for your particular requirements. This is mostly
high-school course with a little bit of lower classes’ course thrown in with a bit of detail here
and there

\section How to read this book?

The usual advice about learning math is to solve problems, hwoever, I will not simply tell that you must
solve the problems. My advice is more detailed. Every chapter will have theory. Read that first. Make sure
you understand that. Of course, you have to meet the prerequisites for the book. Then, go on and try to
solve the problems. In this book, there are no pure problems. Almost all have answers except those which are
of similar kind and repetitive in nature for the sake of practice.

If you can solve the problem then all good else look at the answer and try to understand that. Then, few
days later take on the problem again. If you fail to understand the answer you can always email me with your
work and I will try to answer to the best of my ability. However, if you have a local expert seek his\/her
advice first.

Note that mathematics is not only about solving problems. If you understand the theory well, then you will
be able to solve problems easily. However, problems do help enforce the theory in our mind.

I am a big fan of old MIR publisher’s problem books, so I emphasize less on theory and more on problems. I
hope that you find this style much more fun as a lot of theory is boring. Mathematics is about problem
solving as that is the only way to enforce theory and find innovtive techniques for problem solving.

Some of the problems in certain chapters rely on other chapters which you should look ahead or you can skip
those problems and come back to it later. Since this books is meant for self study answers of most of
the problems have been given which you can make use of. However, do not use for just copying but rather to
develop understanding of the subject.

\section Goals for readers

The goal of for reading this book is becoming proficient in solving simple and basic problems of
trigonometry and geometry. Another goal would be to be able to study other subjects which require this
knowledge like trigonometry or calculus or physics or chemistry or other subjects. If you can solve 95\%
problems after 2 years of reading this book then you have achieved this goal.

All of us possess a certain level of intelligence. At average any person can read this
book. But what is most important is you have to have interest in the subject. Your
interest gets multiplied with your intelligence and thus you will be more capable than
you think you can be. One more point is focus and effort. It is not something new which
I am telling but I am saying it again just to emphasize the point. Trust me if you are
reading this book for just scoring a nice grade in your course then I have failed in my
purpose of explaining my ideas.

Also, if you find this book useful feel free to share it with others without hesitation
as it is free as in freedom. There are no conditions to share it.

\section Acknowledgements

I am in great debt of my family and free software community because both of these
groups have been integral part of my life. Family has prvided direct support while free
software community has provided the freedom and freed me from the slavery which
comes as a package with commercial software. I am especially grateful to my wife, son, daughter,
and parents because it is their time which I have borrowed to put in the book.

To pay my thanks from free software community I will take one name and that is Richard
Stallman who started all this and is still fighting this never-ending war. When I was
doing the algebra book then I realized how difficult it is to put Math on web in HTML
format and why Donald Knuth wrote \TeX\. Also, \TeX\ was one of the first softwares to
be released as a free software. This book has been typepset with \TeX\ using
{\tt Emacs}. I have used macros from the book ``TeX for Impatient'' and modified
them as well as the {\tt eplain}'s macros.

I have used \hologo{METAPOST} for drawing all the diagrams. It is a wonderful
packages and work very nicely.

I would like to pay my most sincere gratitude to my teachers particularly H.\ N.\ Singh,
Yogendra Yadav, Satyanand Satyarthi, Kumar Shailesh and Prof.\ T.\ K.\ Basu. Now
is the turn of people from software community. I must thank the entire free software
community for all the resources they have developed to make computing better. However,
few names I know and here they go. Richard Stallman is the first, Donald Knuth, Edger
Dijkstra, John von Neumann after that as their lives have strong influence in how I
think and base my life on.

I am not a native English speaker and this book has just gone through one pair
of eyes therefore chances are high that it will have lots of errors(particularly with
commas and spelling mistakes). At the same time it may contain lots of technical
errors. Please feel free to drop me an email at \href{mailto:shivshankar.dayal@gmail.com}{shivshankar.dayal@gmail.com},
where I will try to respond to each mail as much as possible. Please use your real names in
email not something like coolguy. If you have more problems which you want to add
it to the book please send those by email or create a PR on
\href{https://github.com/shivshankardayal/Trigonometry-Latex}{github}.
\vskip1.5\baselineskip

\line{\it Nalanda, Bihar\hfil\rm S.\thinspace S.\thinspace D.}
