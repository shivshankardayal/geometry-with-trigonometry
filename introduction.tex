% -*- mode: plain-tex; -*-
\chapter{Introduction}

\pdfoutline goto name {introduction} count 1 {1. Introduction}%
\hldest {}{}{introduction}%
\xrdef{introduction1}

\chapterdef{geometry}

\color{black}Geometry is one of the oldest branches of math along with arithmetic, and one of the most
fundamental. One of the first proofs in Geometry was given by Thales(640 BCE - 546 BCE) of Miletus. It was
proof that a diameter bisects a circle. Pythagoras(570 - 495 BC) of Samos was the most famous among the pupils of
Thales. You might have heard of Pythagorus' theorem which we will study in the chapter of triangles.

The {\it Elements} by Euclid of Alexandria, a Greek matheatician c. $300$ BC, which is a series of $13$
books, is one of the first profound works on Geometry. Most of what we will study about geometry in this
book is based on the book from Euclid. Some scholars believe that the {\it Elements} is
largely a compilation of work of earlier Greek mathematicians. Geometry comes from an ancient Greek word
meaning `land measurement'. It consists of two words `geo' which means earth and `meterin' which means to
measure. Geometry has been studied since ancient times. Most civilizations including that of Babylon, China,
Egypt, Greece and India studied it with varying interest and success. As we say that need is the mother of
invention so the fundamental problems faced by these civilization contributed to the development of
Geometry.

What we will study in this book lays the fundamental principles of basic
geometry, which will be useful to you when you study more advanced branches of mathematics.

\section{Euclid's Definitions, Axioms and Postulates}

\pdfoutline goto name {edap}  {1.1 Euclid's Definitions, Axioms and Postulates}%
\hldest {}{}{edap}%
\xrdef{edap1}

Euclid has given $23$ definitions, $5$ postulates, and $5$ common notions to start with in his first
book. I am presenting these for the sake of posterity.

\vskip0.5cm
{\bf Definitions}

\numberedlist
\li A point is that of which there is no part.
\li And a line is a length without breadth.
\li And the extremities of a line are points.
\li A straight-line is (any) one which lies evenly with points on itself.
\li And a surface is that which has length and breadth only.
\li And the extremities of a surface are lines.
\li A plane surface is (any) one which lies evenly with the straight-lines on itself.
\endnumberedlist

 \endchapter