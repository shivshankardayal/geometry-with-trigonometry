% -*- mode: plain-tex; -*-
\chapter{Geometry}

\pdfoutline goto name {geometry} count 0 {1. Geometry}%
\hldest {}{}{geometry}%
\xrdef{geometry1}

\chapterdef{geometry}

\color{black}Geometry is one of the oldest branches of math along with arithmetic, and one of the most
fundamental. The {\it Elements} by Euclid, a Greek matheatician c. $300$ BC, which is a series of $13$
books, is one of the first profound works on Geometry. Most of what we will study about geometry in this
book is based on the book from Euclid. Some scholars believe that the {\it Elements} is
largely a compilation of work of earlier Greek mathematicians. Geometry comes from an ancient Greek word
meaning `land measurement'. What we will study in this book lays the fundamental principles of basic
geometry, which will be useful to you when you study more advanced branches of mathematics.

\section{Points, Lines and Planes}

\hldest {}{}{lines}%
\xrdef{lines1}

\noindent A {\it point} is a dimensionless figure. This means that it has no length, width and depth. Essentially, a
point is invisible. However, we usually denote it with a .(dot). If we join two points such that resulting
figure has only one dimension of length, and the length of line is equal to the shortest distance between
the two points then the resulting figure would be a {\it line}.

\parshape=1
2.8cm 10cm
\hbox to 0pt{\vbox to 0pt{%
\lfig{sl}{images/line.pdf}{A Line}{2.5cm}
\vss}\hss}
\indent The \ref{sl} shown is a figure of a straight line. A line in general is also called a straight line,
which is the shortest distance between the two points.

On any straight line there are infinite points(or between any two points there are infinite points). A line
stretches from $+\infty$ to $-\infty$. Usually a line
is drawn with arrows at both ends, which denote that the line stretches to infinity. If there are two points
on the line $A$ and $B$ then the line is also written as $\overleftrightarrow{AB}$, and read as line $AB$. A
{\it line segment} on the other hand is of finite length, and is drawn without arrows. A line segment, for
example, in the figure can start at $A$ and end at $B$. Such a line segment is written as $\overline{AB}$ or
just $AB$. $A$ and $B$ are called {\it endpoints} of the line segment $AB$.

\parshape=1
2.8cm 10cm
\hbox to 0pt{\vbox to 0pt{%
\lfig{ray}{images/ray.pdf}{A Ray}{1.6cm}
\vss}\hss}
\indent The \ref{ray} shown is a figure of a ray. A {\it ray} has one endpoint and it stretches to infinity
on other side. Thus, you can split a line into two rays. A ray is written as $\overrightarrow{AB}$.\

Note that $\overrightarrow{AB}$ is not same as $\overrightarrow{BA}$. However, a line segment $AB$ is same as
line segment $BA$.

If you rotate a line by $180^\circ$(rotate in such a manner that ends occupy other end's place), then you
will get a {\it plane}. A plane has two dimensions, length and
width, both of which are infinite. For example, top surface of a table or a page of this book(assuming the
page has zero thickness) are planes. In this book we will restrict ourselves to planar geometry i.e. $2$D or
two-dimensional geometry.
