% -*- mode: plain-tex; -*-
\input amstex
\input eplain
\beginpackages
  % |url.sty| provides the |\url| command which we will use to typeset
  % a URL.  We request that |url.sty| be the version from June~27,
  % 2005, or later, because earlier versions had problems interacting
  % with plain \TeX.
  \usepackage{url}[2015/06/27]
  % |color.sty| provides support for colored text; all hyperlinks are
  % automatically colored by Eplain when this package is loaded.  We give
  % the |dvipsnames| option because we want to use named colors from the
  % |dvips| graphics driver.
  \usepackage[dvipsnames]{color}
  % Finally, we load |graphicx.sty|, for the macros |\includegraphics|
  % and |\rotatebox|.
  %\usepackage{graphicx}
\endpackages
\enablehyperlinks
\hlopts{bwidth=0}

\definecolor{mycolor}{rgb}{0, 0.45, 0.63}
\definecolor{black}{rgb}{0, 0, 0}
\font\titlefont=cmss10 at 40pt
\font\subtitlefont=cmss10 at 25pt
\font\authorfont=cmr10 at 18pt
\font\chaptitlefont=cmr10 at 10pt
\font\partfont=cmr10 at 12pt
\font\sectionfont=cmr10 at 9pt

\pdfpagewidth=210mm
\pdfpageheight=297mm
\paperwidth=210mm
\paperheight=297mm
\input macros

\pageno = -1
\begingroup
  \nopagenumbers
  \headline={\kern-2in\smash{\color{mycolor}\vrule width 100cm height 100cm depth
30cm}}
  ~~~
  \vskip 2cm
  \titlefont\centerline{\color{white}Geometry with
Trigonometry}
  \vskip 2mm
  {\color{white}\hrule}
  \vskip 2mm
  \subtitlefont\raggedleft{\color{white}A problem-oriented approach}
  \vskip 1.5cm
  \cfig{cp}{images/nine-point-circle.pdf}{}{10cm}
  \vfill
  \authorfont\color{white}Shiv Shankar Dayashru
  \color{black}
  \vfil
  \eject
\endgroup
\color{black}
\rewritetocfiletrue % always do a table of contents
\completebooktrue
\edgetabsfalse
% Copyright page.
%
\noheadlinetrue
\sinkage

\bigskip

\bigskip

\vskip 2in
\noindent Copyright \copyright{} 2024 Shiv Shankar Dayashru.

\bigskip
\noindent Permission is granted to copy, distribute and/or modify this
document under the terms of the GNU Free Documentation License, Version
1.3 or any later version published by the Free Software Foundation; with
no Invariant Sections, with no Front-Cover texts, and with no Back-Cover
texts.  A copy of the license is included in the chapter entitled ``GNU
Free Documentation License''.

\medskip\noindent
Under the terms of the GFDL, anyone is allowed to modify and
redistribute this material, and it is our hope that others will find it
useful to do so.  That includes translations, either to other natural
languages, or to other computer source or output formats.

\medskip\noindent
In our interpretation of the GFDL, you may also extract text from this
book for use in a new document, as long as the new document is also
under the GFDL, and as long as proper credit is given (as provided for
in the license).
\vfil
\eject

\noheadlinetrue
\sinkage
\centerline{\it Dedicated to my family.}
\def\xref#1{\xrefn{#1}}

\vfill\eject
\noheadlinetrue
\pagebreak

% Contents.
%
% We never want to empty the file after doing the brief contents.
%
%\rewritetocfilefalse
%

%\ifcompletebook \global\rewritetocfiletrue \fi

\pdfoutline goto name {contents} count -0 {Table of Contents}%
\hldest {}{}{contents}%

\frontchapter{Contents}

%\contents
\begingroup
\raggedleft
\chaptitlefont\hlstart{name}{}{preface}Preface \hlend\dotfill\hfil\xref{preface1}
\vskip1mm
\partfont\hlstart{name}{}{part1}I Geometry \hlend\dotfill\hfil\xref{part11}
\vskip1mm
\chaptitlefont\hlstart{name}{}{geometry}1.\ Geometry \hlend\hfil\dotfill\xref{geometry1}
\vskip1mm
\sectionfont\hlstart{name}{}{lines}1.1 Points, Lines and Planes \hlend\hfil\dotfill\xref{lines1}
\vfill
\endgroup
\pagebreak

\noheadlinetrue
\pdfoutline goto name {preface} count -0 {Preface}%
\hldest {}{}{preface}%
\xrdef{preface1}

{\input preface }

\pagebreak
\settabdepth{\bigtab}
\noheadlinetrue
\pageno=1
\begingroup
\topglue 0pt plus 1fill
\pdfoutline goto name {part1} count -1 {I Geometry}%
\hldest {}{}{part1}%
\xrdef{part11}
\titlefont\centerline{\color{mycolor}Part I}
\vskip1cm
\titlefont\centerline{\color{mycolor}Geometry}
\vfill
\eject
\endgroup

\input geometry


\backmatter
\input fdl

\bye
